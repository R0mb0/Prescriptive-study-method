	\documentclass[11pt,a4paper]{article}
	\usepackage[italian]{babel}
	\usepackage[utf8]{inputenc}
	
	% in way to comment a block
	\long\def\/*#1*/{}
	
\begin{document}
\title{Metodo di studio prescrittivo a cascata.}
\author{Francesco Rombaldoni}
\date{}
\maketitle
\newpage

\section{Concetti preliminari}
 \subsection{Obiettivo del documento. }
Il documento si pone come obiettivo quello di proporre una strategia di studio (da personalizzare a seconda dei casi) seguendo un modello prescrittivo a cascata con parti iterative.\\

\subsection{Obiettivo Fondamentale.}
$\rightarrow$ Si definisce "Obiettivo fondamentale" tutto ciò che soddisfi questi tre concetti: 1)  che sia un qualcosa in grado di discretizzare un problema (come per esempio la divisione in capitoli di un libro). 2) Che rappresenti la conclusione di un argomento (per fare in modo che non ci siano due o più capitoli che trattino di due argomenti simili). 3) Che sia misurabile e non valutabile (i capitoli di un libro letti in una unità di tempo sono misurabili, le nozioni acquisite in una unità di tempo sono invece valutabili).\\

\subsection{Tempo.}
$\rightarrow$ Il tempo può essere considerato come una risorsa tangibile, per tanto esso può essere in eccesso, in difetto ed accumulabile.\\
$\rightarrow$ l'unita temporale normalmente utilizzata è l'ora, essa è composta da 45 minuti di "Sprint" ovvero una fase di sforzo intenso e da 15 minuti di "Cool Down" ovvero un momento di pausa all'interno della quale è consigliato abbandonare la postazione di studio per fare del movimento. \\
$\rightarrow$ Per quanto riguarda l'amministrazione del tempo vengono consigliati due modelli, il primo è il  modello arbitrario; prevede che dopo 45 minuti di studio intenso ci si fermi arbitrariamente interrompendo il lavoro. Il secondo modello prevede invece di interrompere lo studio solo dopo il raggiungimento di un obiettivo fondamentale; in questo caso si consiglia di conteggiare il tempo di "Sprint" fatto e di conseguenza calcolare il tempo di pausa accumulato. (Per esempio dopo due ore di studio consecutive si sono accumulati 30 minuti di pausa).\\
$\rightarrow$ Il tempo di lavoro giornaliero può essere deciso sulla base del carico di studio e delle previsioni fatte, in particolare si consiglia di tenere una linea massima di 7 ore al giorno e una linea minima di 5 ore al giorno. Si vuole far riflette di come 7 ore per giorno si possono dividere come 3 ore di mattina (se fatte consecutivamente diventa una sessione di studio da 2,25 ore di "Sprint") e 4 ore il pomeriggio (se fatte consecutivamente diventano 3 ore di "Sprint").\\

\subsection{Lo studio come un problema.} 
$\rightarrow$ Si definisce problema tutto ciò che per essere risolto impone l'assunzione di un adattamento particolare, volto al raggiungimento di un obiettivo. Lo studio ricade pienamente in questa definizione e per tanto può essere visto come un problema.\\
Il modo migliore per risolvere un problema è suddividerlo in una serie di piccole iterazioni da eseguire fino al raggiungimento dell'obiettivo. Anche se il modello suggerito è prescrittivo ed a cascata, questo non vieta di agire ricorsivamente all'interno di un macro passaggio contenuto nel paradigma a cascata.\\
Riassumendo noi stiamo dividendo un grande problema come quello dello studio in dei piccoli problemi, i quali a loro volta verranno divisi ulteriormente i dei problemi ancora più piccoli di nauta iterativa.\\ 

\subsection{Strumenti}
$\rightarrow$ Gli strumenti consigliati sono: un pc, un blocco di carta e delle penne di colore diverso con cui poter scrivere.\\
 I colori delle penne dovrebbero avere un significato, in particolare si consiglia di usare il colore Nero per scrivere le cose "base" ovvero tutto quello che non ha bisogno di essere messo in risalto, in Blu le cose importanti degli appunti (come per esempio concetti rilevanti, nomi o sigle), in Verde gli accorgimenti (quelle osservazioni personali che potrebbero aiutare a ricordare con facilità gli argomenti), in Rosso Le cose molto importanti ed i titoli dei vari argomenti.\\
\newline

\section{Il metodo}
\subsection{Raccolta di risorse}
$\rightarrow$ Nella fase di raccolta delle risorse bisogna ricercare ed accumulare ordinatamente tutto quello che possa essere d'aiuto alla comprensione dell'argomento comprese le risorse umane, come un amico/a che ci possa aiutare.\\
$\rightarrow$ Uno dei modi migliori per compiere questa fase è quello di avere quanto più possibile materiale utile allo studio in formato digitale, in modo da poter usufruire di tutti i vantaggi messi a disposizione dai programmi moderni oltre che poter risparmiare spazio nell'area di lavoro. In questo caso i file dovrebbero essere stoccati ed ordinati all'interno di una cartella posta in un posto sicuro. (Un posto sicuro può essere all'interno di un hard-disk diverso da quello di sistema.)\\
I file all'interno di questa cartella dovrebbero essere poi suddivisi (creando ulteriori cartelle) in base all'importanza o in base al loro uso, in modo da ridurre il tempo di ricerca della risorsa interessata, per far si di risparmiare tempo nelle operazioni sistematiche.\\ 
$\rightarrow$ I nomi dei documenti e la loro funzione devono essere segnate in maniera ordinata (quindi che permetta veloce lettura) sul blocco di carta come tutte le altre risorse utili allo studio.\\
$\rightarrow$ Essendo anche il tempo una risorsa è necessario comprendere bene il tempo che si dispone per lo studio. Una pessima stima del tempo disponibile potrebbe provocare successivamente delle pessime valutazioni, aumentando così la probabilità di fare "Crouching" (Studiare tutti di fretta poco prima della data d'esame).\\

\subsection{Analisi del materiale raccolto}
$\rightarrow$ Nella fase di analisi l`obiettivo è quello di scrivere sempre sul blocco di carta, una stima del calendario che si pensa di tenere, ovvero una stima del ritmo di studio in relazione ai dati raccolti.(Per esempio i giorni di studio prima di poter sostenere l`esame)\\
$\rightarrow$ Il documento che si andrà a scrivere sarà quindi la stima dello sforzo richiesto per lo studio della materia.\\
 Durante lo studio una parte fondamentale del lavoro sarà quella di segnarsi per ogni giorno lavorativo gli obbiettivi raggiunti, in modo da poter correggere la stima fatta. In questo modo oltre che a prendere coscienza del effettivo rendimento, si avrà anche un valido strumento per valutare la possibilità di aumentare oppure ridurre il ritmo di studio.\\

\subsection{Prendere coscienza degli argomenti}
$\rightarrow$ Nella fase della presa di coscienza degli argomenti l'obiettivo è quello di leggere ogni risorsa scendendo nel dettaglio cercando di scomporla in delle componenti più piccole e maneggevoli.\\
$\rightarrow$ Riassumere ogni argomento al pc cercando di inserire quanti più concetti chiave possibili, in modo da ottenere del materiale che sia più maneggevole per ripassare.\\
$\rightarrow$ Durante la fase riassuntiva scrivere sul blocco di carta la mappa concettuale del argomento, la mappa risultante non dovrà essere più grande di un foglio (Questa cosa serve per inquadrare meglio le parti principali e le loro relazioni).\\
$\rightarrow$ Se per l`argomento corrente non si hanno esercizi o domande su cui fare pratica, scrivere a pc quante più domande (con risposta) possibili riguardo il suddetto.\\
$\rightarrow$ Scrivere sul blocco di carta per ogni argomento analizzato il tempo impiegato, verificando che il tempo speso nell'analisi dell'argomento sia conforme alle stime.\\
 Nel caso in cui s'impieghi più tempo rispetto a quello previsto, aggiornare le stime verificando che sia possibile rispettare la scadenza.\\

\subsection{Macro visione}
$\rightarrow$ Nella Fase di macro visione l`intento è quello di ricomporre le componenti della fase precedente, in maniera tale di poter apprendere i rapporti tra gli argomenti oltre che delle componenti individuate.\\
$\rightarrow$ Comporre le mappe concettuali fatte precedentemente facendo in modo di sottolineare i rapporti tra gli argomenti (per facilitare questa operazione usare i riassunti fatti). L'intento ultimo è quello di ottenere una specie di "mappa" integrale che faciliti la visione d'insieme di tutti gli argomenti. \\

\subsection{Curare l'esposizione}
$\rightarrow$ Nella fase di cura dell'esposizione si fanno degli esercizi per valutare la preparazione raggiunta.\\
$\rightarrow$ Nel caso in cui il compito sia di tipo scritto cercare di fare quanti più esercizi possibili, segnandosi su un foglio tutti i dubbi e gli errori commessi. Per risolvere i propri dubbi oppure per capire meglio la natura dell'errore commesso usare le mappe concettuali per trovare in breve tempo l`argomento, quindi usare i riassunti per risolvere il dubbio o il problema.\\
$\rightarrow$ Nel caso in cui il compito fosse di tipo orale, rispondere alle domande oralmente o scrivendo le risposte su un nuovo documento al pc oppure sul blocco di carta. Confrontare quindi le risposte date con le risposte giuste, segnandosi gli errori commessi e se possibile il motivo per cui si ha sbagliato.\\
$\rightarrow$ Scrivere sul blocco di carta il tempo impiegato per risolvere ogni esercizio o per rispondere alle domande, in modo da poter verificare in ogni momento il rendimento di studio tenuto e per possedere una stima delle prestazioni che si terranno durante l'esame (qualora quest'ultimo sia di tipo scritto)\\

\subsection{Conclusione}
$\rightarrow$ Nella fase di conclusione l'obiettivo è quello di valutare se le conoscenze acquisite siano sufficienti per sostenere l'esame. \\
$\rightarrow$ Valutare se i tempi registrati nella fase precedente sono compatibili con i tempi dell'esame.\\
$\rightarrow$ Riguardando gli errori segnati distinguere gli errori veniali da quelli di concetto; verificare se gli errori concettuali si distribuiscono su tutti gli argomenti o se si concentrano solo su alcuni.  Nel primo caso si potrebbe valutare di non dare l'esame e cercare di fare più esercizi, nel secondo caso valutare l`importanza degli argomenti attorno ai quali si concentrano gli errori. Se si ritiene che l'importanza degli argomenti non sia marginale, allora, come prima, prendere in considerazione di non dare l'esame a favore di più esercizio.\\
$\rightarrow$ Nel caso di errori veniali, se possibile capire da che cosa sono dovuti prima di sostenere l'esame.\\

\section{Ultime considerazioni}
Come preannunciato questa guida offre delle linee guida per la preparazione scolastica. Queste linee guide non sono da considerasi come delle regole da seguire, ma più come uno strumento da personalizzare sulla base delle proprie esperienze e personalità. \\
	
\end{document}